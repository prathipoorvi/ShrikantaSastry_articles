\documentclass{book}

%\font\kan=lelr7t at 12pt

\begin{document}

\chapter*{Karnataka in Legend and History}

{By Dr. S. Srikantha Sastri}

On the auspicious day of the Festival of Lights (Dipavali), the New
Mysore or Karnataka State will come into being, in fulfilment of the
long cherished hopes and aspirations of two crores of Kannadigas. New
Mysore will comprise most of the Kannada-speaking areas, though not
all. The ancient limits of Karnataka, according to the \textit{Kavi
  Raja Marga} of the Rashtrakuta emperor Amogha Varsha, Nripatunga,
extended from the Kaveri to the Godavari in the ninth century. The
purest Kannada was spoken in the area between Purigere or Lakshmeswar
in the Dharwar district, Pattada Kisuvolal or Pattadakal in the
Bijapur district and Koppana or Kopbal in the Raichur district. From
about the eleventh century Marathi and Telugu in the north and the
east and Tamil and Malayalam in the south began to circumscribe the
Kannada region.

According to the States Reorganisation Act, the New Mysore State along
with the bilingual State of Bombay will form the Western Zone. For
administrative convernience, the New Mysore State is tentatively
divided into four divisions. The Mysore division will include the
districts of Mysore, Coorg, Mandya, Hassan, South Kanara, Chikmagalur
and Shimoga. The Bangalore division includes Bangalore, Tumkur, Kolar,
Chitradurga and Bellary districts. The Belgaum division will have
Belgaum, Dharwar, Bijapur and North Kanra districts. The Gulbarga
division includes Gulbarga, Raichur and Bidar districts with modifications.

In the two thousand years of the history of Karnataka, the Kannada
area had been brought under the suzerainty of a single dynasty only in
the times of the Kadambas of Banavasi, the Chalukyas of Badami and
Kalyani, the Rashtrakutas of Malkhed. Even the Vijayanagara empire
did not eliminate the Bahamani kingdoms, north of the Tungabhadra and
the Krishna. Haidar and Tipu made inroads into north Karnatak at the
expense of the Nizam and the Marathas, but no lasting effect was
achieved. The British for their Strategic purposes carved out the
Kannada territory and distributed it among the indian States and the
provinces. The agitation for the unification of the Kannada areas
during the last half a century has resulted in the formation of Mysore
in an Independent India, with an area of 72,730 square miles and a
population of 19 millions.

Mysore, which gives the name to the division and the State, can claim
a legendary antiquity as the place where Mahishasura was vanquished by
the Goddess Chamundesvari. Mahisha Mandala, Mentioned in the Maha
Bharata and the Buddhist works, has been identified with Mahishmati on
the Narmada by some. But in the early Tamil literature, a chief
Erumaiuran is mentioned as ruling near the Nilagiris and
Mahishamandala occurs in grants of the Kadambas of the sixth
century. The Mahabaleswara shrine on the Chamundi hill was a Kalamukha
centre and received grants from Hoysala Vishnuvardhana in the twelfth
century. Nanjangud was a sacred place for the Saivas as early as the
13th century and attracts pilgrims from all parts of South india.

\begin{center}
{\Large Stone Age Cultures}
\end{center}

Pre-historic stone age cultures have been found at Kittur, Gundlupet,
Biligiri Rangan hills, Agara, etc. Srirangapatna is well known as the
capital of the Wodeyars of Mysore from the days of Raja
Wodeyar. Situated between two branches of the Kaveri it was a
formidable place, three miles in length and one in breadth. The palace
of the Vijayanagara Viceroys and the Wodeyars of Mysore existed to the
west of the Narasimha temple built by Kanthirava Narasa Raja. This
ancient palace was destroyed by tipu in 1796. The Gangadhara and
Ranganatha temples go back to the Hoysala times and were later
enlarged by the Vijayanagara and Mysore rulers.

The last desperate struggle against foreign domination by Tipu has
left lasting impressions in and about Srirangapatna. Near the
Paschimavahini Bridge were the posts of major Skelly and
Wallace. General Harris besieged the fortress on 5th April, 1799 and
the final assault was delivered on 4th May. General Baird who had a
knowledge of the palce, reached the outer ramparts in six minutes but
was confronted with another rampart and a wide moat constructed by
Tipu in 1973 after Baird had been released. Luckily the British
discovered a plank connecting the two ramparts and after a hand to
hand fight entered the city forcing Tipu to retreat. To the Water Gate
known as `Hole Diddi Bagilu' Tipu came from the outer ramparts mounted
on a horse with a crowd of fugitives. A deadly volley was pouring from
both ends of the gate and Tipu fell down wounded. His attendants
placed him in a palanquin but he moved out of it and lay among the
wounded. A British soldier tried to seize Tipu's gold belt but Tipu
slashed at him with his sword. The soldier shot Tipu through the head
and Tipu was buried at the Gumbaz in Ganjam by the side of his father
Hyder and mother Sydani Begum. The de Havilland swinging arch (which
collapsed recently), the dungeons, the Sultan's battery, the ruins of
Tipu's palace near the parade ground, the Jama Musjir built over a
Hindu temple, Darya Daulat with its historical paintings, the Lal
Bagh, Kale Gowda's Battery, etc. in Srirangapatna form lasting monuments.


Along the course of the Kaveri there were very ancient places
mentioned by Ptolemy and Pliny in the second century. Pannata and
Punnata were famous for beryl, exported to Greece and Rome. Talakad,
named as Talakata by Vrahamihira in the sixth century was the capital
of the Gangas who ruled the eastern parts of Mysore for about nine
hundred years. The shifting course of the Kaveri has buried the old
city and only five temples have survived. Sivanasamudram, where the
Kaveri takes a mighty leap and Ummattur were conquered by Krishna Deva
Raya from the last Ummattur chief Chikka Ganga Raja. Tadi Malingi,
Marehalli and Malavalli have Vishnu temples built by Raja Raja Chola
and the Tamil inscriptions mentions Sri Vaishnavas long before the
advent of Sri Ramanuja into Mysore. At Somanathapura is the famous
Hoysala temple built by Somanatha, a general of Narasimha III
(1254-1291). Built on a star shaped terrace, three feet high, the
temple follows the star pattern and had three images, beautifully
carved, of Kesava, Janardana and Venugopala. The Sculptures on the
walls and rows of large images have elicited high praise. 

Coorg or Kodagu is mentioned in the early Tamil literature and famous
for its place of pilgrimage Tala Kaveri where the Kaveri takes its
rise. Included in the ancient kingdoms of the Chengalvas and
Kongalvas, Kodagu was conquered by the Hoysalas. Tipu  conquered it
and recruited the hardy people in to the Mysore army.

\begin{center}
{\bf Sri Ramanuja}
\end{center}

Mandya was known as Mantheya and now the Kollegala taluk has been
transferred to the Mandya district. Kollegala near the Kaveri is
associated with Kahola Galava and its silk weaving was famous in the
Vijayanagara days. Maluru, near Maddur, is famous for its Vishnu
temple built by the Chola general of Rajaraja, Aprameya. Under the
Hoysalas this district became famous for the propagation of the
philosophy of Sri Ramanujacharya. Sri Ramanuja is said to have come
from the Chola country to escape persecution, by way of the Nilagiris,
passing through Mirale and Saligrama to Tondanur where Vishnuvardhana
is alleged to have been converted. At Yadugiri or  Melukote Sri
Ramanuja stayed for over twenty years and consecrated the Aprameya or
Cheluva Raya temple. The Narasimha temple at the same place was
constructed by Raja Wodeyar of Mysore (1578-1617). Chikka Devaraya was
a great aevotee and composed Gitas on Cheluva Raya. There are many
Hoysala temples of great artistic merit at Mandya, Basral,
Nagamangala, Kambada Halli, Hosaholalu, Yedatore and other
places. Hirode was called French Rocks because of the encampment of
the French troops in support of Tipu in the Mysore wara.

Hassan or Simhasanapuri is an ancient centre of worship of the Divine
Mother as Simhasanesvari. The Gangas, Chengalvas and Kongalavas held
sway until the Hoysala chief Vinayaditya moved from Sosevur to Halebid
and made it the capital under the name Dvarasamudra. Illustrious
Hoysala rulers like vishnuvardhana, Ballala II, Narasimha II,
Somesvara and Vira Ballala III dominated the history of South India
from the 11th to the 14th century. The Hoysalas succeeded to the
empire of the Chalukyas of Kalyani in Southern Karnataka, expanded in
the north up to the Malaprabha, and fought with the Yadavas of
Devagiri. From the days of Vira Ballala II, the Hoysala rulers became
the protectors of the Cholas and established their capital at Kannanur
Koppam, opposita to Srirangam, to check the Pandyas. Narasimha II, to
protect the Chola Rajaraja II from Perunjinga, started from
Dvarasamudra, released Rajaraja from prison at Sendamangalam. His
generals Appayya and Samudra Gopayya conquered all the country up to
Ramesvaram.  Vishnuvardhana may be called the Maker of Mysore as it
was before integration.

Hoysala architecture, as typified by the monuments at Belur and
Halebid, was a development of the Chalukyan style; but because of the
greater elaborateness can claim an independent status. The Hoysala
temple may have even five garbha grihas in the same complex and there
are also twin shrines standing side by side built on the same plan. At
Halebid is the ``Parthenon of Hoysala Art''. The Hoysalesvara and
Kedaresvara temples moved Fergusson to ecstasy. ``One cannot help
wondering at the amount of lovely detail, all finished with scrupulous
care tucked away in dark corners, where it is often impossible to sea. But,

``In the elder days of Art,

Builders wrought with greatest care

Each minute and unseen part;

For the Gods see everywhere.''


Belur was also the capital of the Hoysalas and remained as the capital
of the last nominal emperor of Vijayanagara, Sri Ranga. Mythologically
it was associated with the story of Mohini and Bhasmasura and hence
Vishnu as Chenna Kesava has feminine features. Vishnuvardhana and his
Jaina queen Santala Devi constructed the Vijaya Narayana and Kappe
Chenniga temples in the same compound, to celebrate the victory over
the Cholas at Talakad in 1117 A.D. Subsequently Vira Ballala II added
many sculptures and the rulers of Vijayanagara, Mysore, Periyapatna,
Coorg and Belur made considerable alterations and additions. The
conservation and renovation of the whole temple was completed recently
on the initiative of the late ruler of Mysore, Sri Krishna Raja
Wodeyar IV. The temple at Arasikere built by Vira Ballala II is also
one of the finest examples of the Hoysala Style.

Sravana Belagola was known as Katavapra or Kalbappu and is associated
with Mauryan emperor Chandragupta and Sruta Kevali Bhadrabahu. The
monolithic colossal status of Gommata, 58 feet high. is the biggest
free standing statue in the world . The artist has drawn ``from the
blank rock the wondrous contemplative expression touched with a faint
smile, with which Gommata gazes out upon the struggling world''.

South Canara, like the other districts of the west coast, is
associated with Parasu Rama and was ruled by the Alupas. Mangalur and
Barakur on the Netravati were famous ports mentioned by Ptolemy, Pliny
and the Periplus. Conquered by the Hoysalas, it was ruled later by the
Viceroys of Vijayanagara. karkala and Mudabidire are famous centres of
Jainism and at Karkala and Yenur are their monolithic statues of
Gommata, erected by the local rulers. The Keladi kingdom extended over
the Tuluva country and after the fall of Keladi. Hyder and Tipu made
Mangalore a stronghold against the Bombay British government. Udupi
was the centre from which the Dvaita philosophy of Sri Madhvacharya
spread throughout India. The oldest shrines at Udupi are those of
Chandramouli and Anantesvara from where Sri Madhva is said to have
disappeared from mortal ken. The temples of Sri Krishna and Durga,
Pajaka kshetra, the birth place of Sri Madhva, Subrahmanya are other
holy places.

\begin{center}
{\bf Bulwark Against Muslim Attacks}
\end{center}

Chikkamagalur and Hiremagalur are associated with Janamejaya and a
Yupa pillar with an inscription in Brahmi characters of about the
third century, testifies to the existence of Vedic religion. Mudagere
with Sasapura or Sosevur (Angadi) first witnessed the rise of the
Hoysalas, Sala the eponimous hero, who founded the dynasty worshipped
Vasantika Devi of Sasapura. Banavasi or Vaijayanti is mentioned by
Ptolemy as \textit{Buzanteion-a} a famous sea-port. The Kadambas of
Banavasi for more than three centuries wielded power all over the
central and western parts of Karnataka after the Satavahanas. The
Kadambas established a branch in Orissa and the cult of Madhukesvara
was taken from Banavasi to kalinga and thence to South Burma and
Cambodia Gersoppa was a capital of the Tuluva Bhairarasas and a Jaina
centre. Sringeri is traditionally associated with Sri Sankara and
Suresvara. The Jagadgurus of Sringeri. Sri Vidya Sankara, Sri
Vidyaranya and Bharati Krishna Tirtha inspired Harihara and Bukka to
establish the Vijayanagara empire in 1336, as a bulwark against the
Muslim attacks on Hinduism. From the very beginning of the empire, the
emperors expressed their gratitude to the Sringeri Jagadgurus and a
branch of the Sringeri Matha was established in Vijayanagara itself,
near the temple of Virupaksha, the tutelary deity of the Vijayanagara
emperors. The Vidya Sankara temple at Sringeri is a unique monument,
built on the plan of Sri Chakra and apsided like the ancient Chaitya
caves. The Shimoga district possesses some of the oldest dateables
monuments in India at Talagunda, and Mallavalli. The Pranavesvara
temple at Talagunda was considered very holy even by the
Satavahanas. Baligame possessed a great University under the
supervision of the Kalamukha Saiva Acharyas, whose influences extended
all over India. Keladi, Bedanur and Ikkeri




\end{document}
