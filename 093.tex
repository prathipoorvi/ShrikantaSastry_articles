\documentclass{book}

%\font\kan=lelr7t at 12pt

\begin{document}

\chapter*{Inscriptions of Karnataka}

Karnatak Inscriptions Volume 2: Edited by Vidyaraina R. S. Panchamukhi
(Kannda Research Institute, Dharwar. Price Rs. 3).

The inscriptions collected by the search institute during the year
10-1941 had been partly publish in the first volume in 1941 but ??? rest
had to await publication now, due  to the suspension of archaeological
reports until the ??? sation of war. The second lame gives the texts
of forty-two inscriptions, with the notes and translations by the
editor.

The inscriptions belong to the early kadambas, Calukyas of Banavasi,
the rastrakutas, the calukyas of Kalyani, Kalacuryas of Kalyani, the
Rattas Saandatti, the Hoysalas and the davas of devagiri. In the
general reduction the editor refers the reach to his annual report for
the year 40-41, but it would have been more ??? ful if a fuller
discussion had been ??? duded in the present volume, pubshied twelve
years later. The \textit{Bombay-Karnatak Inscriptions, edited by ???
R. Shama Sastri, and Prof. Kunangar's Inscriptions in North ??? Karnatak
and Kolhapur} have brought to the new maierial and a fresh dission
would have been welcomed. a inscription No. 38 has already in
published by Kundangar (No. ) and the editor himself has edited, Badami
inseription of Calukyas ??? llabhesvara in the \textit{Epigraphia
  India}, Vol. 27. p. 4. As one of the esfst ??? records dated in he
Saka era (S. ???1) its palaeography deserves deded ?? alscussion. The
copper-plaient of Kadamba Krsnavarina II ??? nations only his
great-grandfather and not his immediate predeccssors and his summame
Dosarasi??? ragankita ??? his \textit{priya putra} Ravivarma (till to
unknown) have to be discussed detail. The grant was issued from
vijayanti in the fifteenth year but ??? vivarma of the Ranavasi branch
is Krsnavarma's contemporary and great conqueror. It is probable that
?? snavarma had to acknowledge Ravivarma as his ``dcar
son''. Ravivarma ?? an end to the Triparvata branch an after, as some
of his records were used from Uccangi. The Kolhapur ?? aut of
Vinayaditya of Saka 615, 15th ?? ar. was written by the same
Srirama??? Punyavallabha as in the Hariro ?? grant of the fourteenth
year, the references to the king of the musas, the queen of striralya
and Lata 22) need elucidation. Kalingeti 32) is not Kalingapati but
the  ??? form of many proper names in paient Karnataka, probably
kalinga Ganga Madhava mentioned in a ?? grant of Vikramaditya 1
(M.A.R. to No. 30). The editor connects aayaditya with the father of
Arisari of the Kollipara and Parbhani ants. The Kollipara grant
(\textit{Sources Karnataka History,} Vol. 1. p. 143, ??? LXXXIL) is
not reliable as it ?? ams to give the date Kali 6128 (A.D. 67). and
Arikesari, the son of Vinaditya Yuddhamaila, who had the ?? les Sri
Rarne, Nrpankusa, Ranavikarna is called Tribhuvane malla, a ?? Kalyani
Calukya title Rampa's count in the Bharata, the Yaga asaka Campu, the
Kollipara and Parand inscriptions give different verins of this branch
of the Calukyas, ?? Dandapur inscription of Prabhatairsa (I.A. 12,223,
M.E.R.E. 63 of 1934 K. Carite, Vol, r. p. 28) has long an famillas. It
is now generally uppted that Prabhuta Varsa Govinda had been the joint
ruler with his either, Indra Ill, who ruled up to D. 926 (M.E.R. 235
of 1938, S.I.I. No. 65). Srivijaya of the Danavula ?? insaription in
the reign of Indra ?? (0.1. 10,150) may be the saribe this
inscription. The reference to flora (Dhrava) needs
explanation. i.e. ??? nulltary organisation of the solars (Ranta
sasirvar) in No.8 as a sporation should be compared with a
contempotary Cola system. Recording the two new Silanara brannes
(sion. 12, 14 and 15) , they are also earioned in several
insecriptions undangar No. 2. M.E.R.E. in of 04. etc.).

The editor attaches importance to a statement (Nilakanta Sastri. The
has, p. 13) that Kulattunga reached a western sea. But this is
baseiess the Kannda inscriptions prove at Kulottunga had been repulsed
am Talakad as early as A.D. This by anuvardhane and the war was carred
into the Cola empire upto Kanchied Bezavada. The measuring rod???
isnugardhanana ?? gale (No. 20) in sogal in A.D. 1191 is
interesting. The ?? ant of Kalacurya Abavamalla (No. ??) dated
A.D. 1182 and composed by ibbuvada Vidyacakravari! Vijayatyadeva and
written by Pandita viksmldhara??? should have been cornneed with
M.E.R. cp. S. of 1934 commepsed by the same anthors under yava Simhana
(granting Kukkanur ??? year. ind. Ant. 4, p. 274). The kkaikot
inacriptions (Nos. 32 and 41 about A.D. 1288??) quote the sayinga the
saint Siddharama of Sonnalige, of. D. L. Narasimhachar (Introduction
to Siddharama cariteya samaha, ?? 1951, p. XVI) says that the
Man ?? vadi inseription of A.D. 1234 is the allert of the inscriptions
referring the sayings. The inscriptions of the idevas 19. 23, 39 and
41 M.E.R. 1938) ??? is the krants made to Kapiladdha Mallikarjuna of
Sonnaiige. ?? earliest is of Jaitrspals (No. 41, ?? D. 1200-1). Thus
bringing the date Siddharama to the end of the 12th century.

The edition in his notes refers the ader to his general introduction
rerding some of the controversial sues but the introduction merely  ??
ornises to deal with them at e ture date. The verifieation of the
ironational date in the inscriptions aves much to be desired. We
welcome this volume as a useful addition the corpus of Kannada
inscriptions.

\hfill Dr. S. Srikanta Sastri


\end{document}
