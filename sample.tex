\documentclass{book}

\font\kan=lelr7t at 12pt

\begin{document}

\chapter*{Historical Vicissitudes of Bellary}

\begin{center}
By Dr. S. Srikantha Sastri, {\tiny{M.A., D.Litt.}}

\medskip
\bigskip

{\large *}

\bigskip

\begin{tabular}{p{9cm}}
{\kan paramasAvxmijaTAdharaM shashidharaM gaMgAdharaM kAlasaM-}\\
{\kan haranajAcnxnaharaM puratarxyaharaM swKAyxkaraM shaMBu shaM-}\\
{\kan karanAnaMdakaraM samaMtenage Bakitx jAcnxnaveYrAgayxmaM}\\
{\kan karuNaMgeyodxlaviMde haMpeya virUpAkaSxM shivAvalalxBaM}\\[7pt]
\multicolumn{1}{r}{\kan -(harihara)}
\end{tabular}
\end{center}

\bigskip

\centerline{\bf I}
\smallskip

``Like the auspicious {\em tilaka} mark on the fore-head of Jambu-dvipa Lakshmi, is the Tungabhadra rivalling the Ganga of the North, with its banks thickly crowded with the stately trees like Asoka, Punnaga, Amra etc''.\footnote[1]{{\em South Indian Inscriptions} ({\em S.I.I.} IX. 1. 297).}

In this fair land of Kuntala the famous cities of ancient times like Hastinavati (Anegondi --- Kunjara kunja or Kunjara kona), Ujjini (Ujjayani), Kampili (Kampilya) reveal the sacred associations with the historic places of North India, celebrated in the National Epic of India --- the Mahabharata. The holy Pampa lake is sanctified by the divine feet of Sri Rama and Hanuman. The pre-historic dolmens, chromlechs, menhirs, and stone-circles that are found abundantly in the Bellary region show that the stone age cultures of Karnataka had strong affinities with similar cultures in South Africa and the Far Eastern countries, perhaps at a time when the southern peninsula formed a continuous land-mass with Africa and called Gondawana land or Gondaranya. The explorations at Kuppa gallu, Sandur, Ramadurga, Bellary face-rock hill etc., have revealed the existence of stone artifacts of the Palaeolithic, Microlithic and Neolithic cultures of great antiquity\footnote[2]{Bruce Foote. {\em Catalogue of Antiquities}.}. Kishkindha and Pampa of the Ramayana, Hastinavati (Anegondi) of the Maha Bharata, Bhaskara Kshetra of the Puranas, the Bhuvanesvari Pitha of the Tantras and Agamas --- this region is celebrated in song and story, legend and history. Throughout the Epic and Puranic periods this Tungabhadra territory, connected by the river with Sringeri, included Vanavasi (Banavasi), Kuntala, Kampili, Hemakuta, Dorevadi, Edatore nadu etc., was the meeting place of Northern and Southern cultures. Aryan, Dravidian, Munda-Mon-Khmmer and Negrito. Pampa is mentioned in the drama Kaumudi-Mahotsava and in Kannada literature it is identified with the Asrama of Visvamitra and Benares itself.\footnote[3]{Raghavanka : {\em Harischandra Kavya}}

\medskip
\begin{center}
{\bf II}
\end{center}
\smallskip

With the dawn of recorded history we have more definite evidence of the importance of this region. The Nandas of Magadha are said to have ruled Nagarakhanda\footnote[4]{S. Srikantha Sastri : {\em Sources of Karnataka History.} Vol. I.} and when the Mauryas succeeded the Nandas, it formed the southern most limit of the vast Mauryan Empire, as shown by the inscriptions of Asoka at Erragudi, Maski, Kopbal, Palkigundi, Siddhapura, Jatinga Ramesvara, Brahmagiri. The capital of teh southern Viceroy of the Mauryan Empire was at Suvarnagiri and a sub-capital of the governor was at Isila as recorded in the Brahmagiri edict of Asoka.
\begin{quote}
{\em ``Suvamnagirite Ayaputasa mahamatanam ca vacanena Isilasi mahamatanam ca arokiyam vataviye''.}
\end{quote}

There was a crown-prince (Aryaputra) as the Viceroy at Suvarnagiri and he passed on the orders of Asoka to the mahamatra officer in charge of Isila. Suvarnagiri has been identified by some scholars with Kanaka giri or Hemagiri (Hemakuta) but it is more probably Jonnagiri near Erragudi where all the edicts of Asoka are found in the same place. The excavations at Brahmagiri\footnote[5]{{\em M.A.R.} 1942. {\em Ancient India,} Vol. IV.} in the Molakalmuru taluk of the Chitradurga district have revealed the existence of a town of Mauryan times and quite recently Mauryan vestiges have been revealed at Turuvanur. It is probable that Isila was at Brahmagiri. Isila was probably so named after Isipattana or Rishipattana near Benares, famous in Buddhist literature.

\medskip
\begin{center}
{\bf III}
\end{center}
\smallskip

The political and cultural heirs to the Mauryas were the Satavahanas, wrongly called the Andhras or Andhra-bhrtyas in the Puranas and even more wrongly identified with the Telugus. That the Satavahanas were undoubtedly of Kannada origin is proved by their Prakrt inscriptions which mention Satahani-hara or rashtra, identified by Dr. Sukthankar with the Bellary region.\footnote[6]{Hirehadagalli plates of Siva Skanda. {\em E.I.,} VIII ff.} The Satavahanas considered the Srisailam region also as their strong-hold as shown by the account of the Chinese pilgrim It-sing who says ``in the Dekhan the king Sadavaha built a monastery for Nagarjuna at {\em Po-lo-mo-lo-ki-li} (Bhramara giri) and a temple Po-to-yue (Paravata or Parvata), which was a Buddhist shrine, until the Brahmins by a stratagem usurped it''.\footnote[7]{It-sing. tr. Legge.} The views of Dr. Paranavitane and others that, not Srisailam in Kurnool but Nagarjunikonda, where the Ikshvaku inscriptions and Buddhist remains are found represents Srisaila, are extremely improbable. Sriparvata is mentioned in the Chikkulla plates and Talagunda inscription of Kadamba Kakusthavarman.\footnote[8]{{\em E.I.,} VIII.} We also know from the Talagunda inscription that long before c. 450 A. D. the temple of Mahadeva or Pranavesvara of Sthana kunduru was in existence and that Satakarni and other famous kings of yore had worshipped the god and obtained their desires. Dr. N. Venkataramanayya has suggested that perhaps even the name Karnata is derived from the Satakarnis. Karnata is Kannavishava or Kannadu, the original Karnata being the region round. Srisaila, the home of the Satavahanas.

\medskip
\begin{center}
{\bf IV}
\end{center}
\smallskip

Kuntala formed a part of Karnataka and of the empire of the Satavahanas. One of the Satavahana rulers Kuntala Satakarni is mentioned in the Kama Sutras of Vatsyayana in the second century A. D. The commentary says that he was so called because his birth was in Kuntala ({\em Kuntala dese jatatvat}).\footnote[9]{{\em Chakladar} : Studies in the {\em kama Sutras.}} Kuntala Nagara figures in inscriptions and literary works like the Jaimini Bharata and is identified by some with Nubattur, north-west of Mysore. Kuntala desa included in historical times Krishna varna (Krishna), Kurugodu, Gangavadi, Nargunda. Toregal etc. The kavya and {\em Kuntalesvara dautyam} says that the great poet Kalidasa was sent by the Emperor Chandragupta Vikramaditya to the Kuntala court to admonish his grandson, the ruler of Kuntala, who was neglecting the affairs of state in pursuit of pleasure. Many scholars have identified Kuntala with Vidarbha where the Vakataka prince Pravarasena claims to have composed a Prakrt kavya Setubandha, really the work of Kalidasa. But it is also probable that Kuntala represented the Kadamba kingdom, as Kakustha varma claims that his daughters brought prosperity to the Guptas and other kings and the Kadamba kings had names like Bhagiratha, Raghu, Kakutstha, Mandhata etc., showing the influence of the {\em Raghuvamsa} of Kalidasa.

To the west of Kuntala was {\em Kon-ki-na-pu-lo} (Konkanapura) of Yuwan Chwang, probably identical with Banavasi. Banavasi or Vaijayanti was also the capital of the Satavahanas. The Nasik inscription of Gautamiputra Satakarni mentions his kataka at Vaijayanti. {\em Buyzantion} is mentioned by Ptolemy and Vaijayanti mahadvara by Jinasena. The Banavasi Prakrt inscription mentions Sanjayanti, which according to the Mahabharata (II, 31, 70) is near Karahata (Kolhapur). The Kolhapur, Banavasi, Malavalli and the Nagarjunikonda inscriptions show that Kuntala in Karnataka is the original home of the Satavahana family and after Gautamiputra, his son Vasishthiputra established a new city ({\em Navanagara}) at Paithan (Pratishthana) and conquered the Telugu country.

\medskip
\begin{center}
{\bf V}
\end{center}
\smallskip

The successors of the Satavahanas in this region were their provincial governors who belonged to the Pallava and Chutu-Naga dynasties. The Pallavas are first found in Satahanihara (Bellary and Dharwar) as proved by their Prakrt charters.\footnote[10]{{\em E.I.,} VIII, Hirehadagalli C.P.} Later they captured Kanchi from a Saka Satyasimha and extended their domain into the Telugu districts. Even the original Chola country was near Kurnool and Kadapa as shown by {\em Chu-li-ye} of Yuwan Chwang and the Cholas of Renadu.

Under the Kadambas of Banavasi the territory from the Western sea upto Uchangi Durga in the Bellary district constituted ``Banavasi 12000'' country. Uchangi was the capital of an eastern branch of the Kadambas. Mayura Sarma in his Chandravalli inscription\footnote[11]{{\em M.A.R.} 1929. S. Srikantha Sastri : {\em Sources}. Vol. I.} says that he ruled over Sendraka Vishya in Satahanirashtra. The hill fort of Uchangi or Ucha Srungi is of great antiquity marking the limits of the Kadamba and Pallava empires. Sandur with its famous temple of Kartikeya was a holy place even in the days of the Kadambas and Chalukyas of Banavasi and is called Svamimalai or Savimalai. The Chalukyas of Badami conquered Banavasi. The great emperor Pulikesin II had established his sons as the Viceroys in the Bellary and Karnul districts. Adityavarma and Chandraditya ruled here, after the capture of Badami by the Pallavas. Adityavarma made a grant of the villages of Mundakal and Paragere. A recently discovered grant of Abhinavaditya, the son of Adityavarma mentions Netkunda in the Uchangi visaya. The Honnur grant of Vikramaditya I mentions Sagala and Nellikundi, now in the Davanagere Taluk.\footnote[12]{{\me M.A.R.} 1939. 138.} 

\medskip
\begin{center}
{\bf VI}
\end{center}
\smallskip

Under the Rashtrakutas of Manyakheta this region was governed by their feudatories, the Gangas of Talakad. Durvinita in the fifth century and Sripurusha had extended the Ganga kingdom up to the Hagari river.\footnote[13]{S. Srikantha Sastri : {\em The Early Gangas of Talakad.}} Butuga II had obtained Banavasi 12000 country and Nolambavadi provinces from his brother-in-law, the Rashtrakuta emperor Krishna III, for killing the Chola crown prince Rajaditya, the son of Parantaka Chola in the battle of Takkolam. Narasimha Nolambantaka and his general Chavunda Raya captured the fort of Uchangi and Marasimha's death occurred at Bankapura. The temple of Kartikeya was also famous like Manchala (now called Mantralaya) in Adoni, Krishna II (Akalavarsha Subhatunga), the son of Nripatunga, at the request of his feudatory Kanna granted to the goddess Ellamma at Manchala in Sindavadi 1000 in 893 A. D.\footnote[14]{{\em S.I.I.} IX. 1. 55.} The inscriptions of Krishna III mention Vadangili (Hadagalli), and Ramesvara of Proddatur. He was a Saiva and when he visited Srisailam with his queen he made a grant at Jyoti.\footnote[15]{{\em S.I.I.} IX.} An inscription at Ramadurga (Adoni taluk) refers to Duddayya, son of Amogha varsha who made a grant at Indavala near Ujjeni. The Sandur inscription\footnote[16]{{\em S.I.I.} IX. I; {\em M.A.R.} 1935, 49.} found in the Parvati temple on the Kumarasvami hill begins with an invocation to Brahma Skanda. Rattara Meru Ganda-martanda Krishna gave a village named Tataka (Kereyapalli) to Shanmukha. This grant was renewed by the Hoysala emperor Vira Ballala II who was residing at Madhuvana on the bank of the Tungabhadra, at the request of his general Mahadeva. The donee was a devotee of Vishnu, named Vishnukara Brahmachari and the grant is for the worship of the god Sri Svami Deva. This raises very interesting problems of relationship between Vaishnavism and Kaumara cult, The grant is dated 11th March, A. D. 1206, Saturday, Solar eclipse.

Not only the worship of Kumara, Siva and Bhuvanesvari, the worship of Surya was also prevalent here. Nimbapuri was the place of Nimbarkacharya who developed the Dvaitadvaita school of Vaishnavism. Hampi is also called Bhaskara Kshetra in the Vijayanagara inscriptions. Jainism was also prevalent. Though the so-called Jaina temples on the Hemakuta hill have now been proved to be those of Siva, Vishnu etc.,\footnote[17]{{\em M.E.R.} 1939-43. Introduction.} of the Chalukya times, other places nearby like Konakondla, Adavani, Chippagiri were Jaina centres. Konakondla is associated with Kundakunda and has Jaina inscriptions from about the eighth century.\footnote[18]{{\em M.E.R.} 1941. 450-458.}

Through the region was associated with Sri Rama and Hanuman, Vaishnavism is a comparatively late religion here. It was during the fifteenth century that the Vithala temple\footnote[19]{S. Srikantha Sastri. {\em Q.J.M.S.}} and the temples on the Matanga and Malyavanta came to be constructed. When Srivaishnavism and Vaishnavism of Madhva influenced the Salva, Tuluva and Aravidu dynasties by the efforts of the Tatacharyas and Vyasa Raya numerous temples of Rama, Achyuta, Hanuman etc., were constructed.\footnoot[20]{S. Srikantha Sastri. {\em The Vijayanagara Sex-Centenary Commemoration Volume.}} 

Vira Saivism's prominence came in the fifteenth century when the court of Deva Raya II was the scene of the activities of the Nutanas --- Lakkanna, Jakkanna, Maggeya Mayideva, Kumara Bankanatha and others.\footnote[21]{S. Srikantha Sastri : Deva Raya II. {\em 1.A.,} 1929.} In the last quarter of the fifteenth century, Tontada Siddhesvara and his disciples established their mathas and Virupaksha Pandita of {\em Chenna Basava Purana} in the sixteenth century has given an account of the spread of Vira Saivism.

\medskip
\begin{center}
{\bf VII}
\end{center}
\smallskip

The Western Chalukyas of Kalyani had made Kampili on the Tungabhadra as one of their capitals. It figures as the scene of battles between the rival empires of the Chalukyas and Cholas. Raja Raja I and Rajendra I in the wars against Tailapa II and Satyasraya frequently overran this territory. The Chalukya feudatories, the Nolambas of Uchangi were ruling over Arakere and other places in the Bellary district.\footnote[22]{{\em S.I.I.} IX. 1. 42-43.} Ahavamalla Tailapa gave a grant in 996 A. D. to Kartikeya Tapovana.\footnote[23]{{\em S.I.I.} IX. 1. 78.} Kampili is mentioned in an inscription of Jayasimha II in 1018 A. D.\footnote[24]{{\em S.I.I.} IX. 1. 78.} Arasiyakere, Mosangi, Pottalakere, Puvena Padangili were subsidiary camps of the Chalukyas. In the wars between Somesvara I and the Cholas, he killed the Chola King Rajadhiraja I in the battle of Koppam. The brothers of Rajadhiraja, Rajendra II and Vira Rajendra claim in their Tamil inscriptions, to have avenged the defeat by routing Vikramaditya VI and burning Kampili. Kampili was thus a very important outpost of Karnataka from at least the eleventh century. Under Vikramaditya VI this territory came to be called Dorevadi. An inscription of 1099 A. D.\footnote[25]{{\em S.I.I.} IX. 1. 166.} mentions a village near the highway of Dorevadi given to Svami Kartikeya tapovana in charge of Dandasani Visvarupa Brahmachari and Sajavi Chakrapani of Svami Devara pattanakottitone. Siriguppa is called Sirivura in Ballakunde 300.\footnote[26]{{\em S.I.I.} IX. 1. 167. 1102 A. D.} Adavani 500 and Tumbula were included in Sindavadi.\footnote[27]{{\em S.I.I.} IX. 1. 172.} Beluruvinanelevidu of Uchangi Pandyas is mentioned as in Nolambavadi.\footnote[28]{{\em S.I.I.} IX. 1. 215.} Ballareyabidu occurs in an inscription of Vikramaditya VI in 1118 A. D.\footnote[29]{{\em M.E.R.} 1936. E. 64.} but it does not seem to be the present town of Bellary but Bellare or Balambidu in the Hangal tuluk of Dharwar. Perhaps the first definite mention of Bellare occurs in a inscription of 1141 A. D.\footnote[30]{{\em S.I.I.} IX. 1. 237.} The Kuditini and Kolagallu inscriptions of Krishna III of 947 and 984 A. D. refer to the tapovana of Svami Kartikeya and Samadhigata pancha mahasabda Svami Kartikeya tapovanadhipati Lokeya Gavunda, a brahmana and Kartikeya tapovanadhipati Gajadharayya of Kolagal and to the matha of Satyarasi in that place.\footnote[31]{{\em S.I.I.} IX. 1. 65, 67.} In 1148 A. D. Jayasimha III gave a grant to Svami Kartikeya tapovana Dandasani Lohasani Madhava Brahmachari at Kottitone to feed the pilgrims ({Yatrege banda kappadi}). The Oruvay inscription of 1148 A. D. refers to Dorevadi nadu.\footnote[32]{{\em S.I.I.} IX. 1. 353.} 

\medskip
\begin{center}
{\bf VIII}
\end{center}
\smallskip

The Hoysala emperor Vishnuvardhana defeated the Cholas and Chalukyas and then invaded Banavasi and Nolambavadi. He spent his last years at Bankapura. Vira Ballala II, as already mentioned was at Madhuvana near Sandur. After defeating Yadava Bhillama at Soratur, he invaded the country upto the Ghataprabha and Gadag. He had reduced the fort of Uchangi and his queen Padumala was ruling Posavadangile when he had his camp at Hallivura and Lokkigundi in 1212 A. D.\footnote[33]{{\em S.I.I.} IX. 1. 330.} Another inscription of his mentions Kurugodu Kampila sthala in 1217 A. D.\footnote[34]{{\em S.I.I.} IX. 1. 334.} In 1219 A. D. Ballala gave a grant to Svami Kartikeya tapovana Dandasani Lohasani Udayakaradeva and established Telligi.\footnote[35]{{\em S.I.I.} IX. 1. 336.}  






\end{document}
