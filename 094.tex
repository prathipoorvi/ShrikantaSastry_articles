\documentclass{book}

%\font\kan=lelr7t at 12pt

\begin{document}

\chapter*{Karnataka Music}

\begin{center}
The Story of its Evolution
\end{center}

(By Ms. S. Srikanta Sastri M. S. Mysore University)


The broad division of Indian music into Hindustani and Karnataka types
that obtains at present amongst the writers on the subject is capable
of further sub-division from the evolution view-point. The main steam
of musical tradition that originated from Bharata underwent many
changes with the changing years. New influences were brought to bear
upon the two branches of the main stream so that each acquired a
characteristic peculiar to the cultural tradition of its part of the
country. It is a long way from the classic purity and retrained
beauty of Dhrupad to the imaginative intensity of the Thumri in the
history of North Indian music which was subjected to Arabian and
Persian influencess. But South India here as elsewhere stands to-day
as the rigid custodian of more or less traditional forms. Not that
there were absolutely no changes whatever, for music was always a
``live'' art and is still so in spite of the scant attention it
receives from the so-called educated, classes. No vital art can be
static and so we find that certain changes and improvements were
effected from time to time.

Karnataka culture in this as in other branches of its activities
presents a synthesis of forms and symbols of apparently alien
cultures. It reconciles the northern Samskrita tradition with the
Dravidian forms of the far south. Modes indigenous to the country were
assimilated into the scientifically evoluted forms of the north in
such a skilful way that it is usual to find in ordinary parlance an
impression that mathematical rigidity of form ebaraeterises.
Karnataka music. It is a symbol of Karnataka culture and  presents
picture of involution subjective in its criteria, introspective in its
media and emotive in its materia. It stands apart even as the Hoysala
temple does from the huge gepurams of the South and the slender
minarets of the Taj.

\begin{center}
The Samskrita Influence
\end{center}

In the early stages the Sainskrita inflnence was apparent in the
chanting of the vedie hymns. Even among the Sama hymners of Senthern
India three or four distiner schools are clearly discernible. Further,
while the Pada and Krama Pathas of the Bks as aide to memory were
common to all India, Jatapatha and Ghana Patha seem to have been the
monopoly of the South. Thongh not so elaborate in music scales as the
Sama hymns yet Jata was something of a favourite and recited on festal
occasions. From the Udatta Svarita and Anndatta grew the heptatonic
scale of Sama  chants Prathama, dvitiya, tritiyu, chaturthe, mandadi,
Svarthakha and Utkrisbta though these do not necessarily equate with
Shadja, Kishabha, Gandhara, Madhyama, Penchama, Dhaivata and
Nishradha. Mr. Frox Strangways believes that Karnataka Sama scale
corresponds to the B to b of the white notes of the piano, the mixed
Lydic of the Greek mode and the Docrian of the ecclesiastical.

\begin{center}
\textbf{Indigenous Modes}
\end{center}

Of the indigenous modes we have no exact information. There must have
been some, for we find numerous allusions in literature and epigraphs
to the fact that Kannadigas were naturally a musical race. Rajasekhara
(c 900) tells us that Kannadigas used to recite poetic compositions in
a sonorons, sing-song fashion. Nripatunga Amoghavarsha, the
Rashtrakuta emperor (c 375) says that the language of music came
naturally and spontaneonsly to the people. His guru and older
contemporary Jinasena in his Mahapurana gives a description of acting
good music of his time. Saigotta Sivamara (c 850) the Ganga King is
described in inscriptions as well-versed in music and dramaturgy and
as the author of a Gajashtaka which was so popular that it was sung by
women while husking paddy. The capital of Kadamba Kakusthavarman (c
400 A. D.) is said to have resounded with music at an hours of the
day. Pallava Mahen ?? ?? ?? ?? ?? ?? and painter but also a
musician. HIs Kudimiyamaiai inscription  gives the symbols of Rishnbha
as Re and Ru the exact values of which it is hard to  determine. Early
Sangam literature (c 300 A. D.) mentions a species of song and dance
that was executed by Kanndigas about the region of the Nilagiris. In a
Greek drama of the and century A. D. found in the Oxhyryncus Papyrii a
``barbarie'' song and dance executed by the King of Malpi is
mentioned. 

Narada seems to have been the earliest expounder of the Bharatiya
school of music. ``Gandharvakalps'' a work at tributed to him, is said
to be an authority on Sama Game. ``Naradiya Siksha,'' a late work
probably of the 10th century expounds the saptasvaras three gramas, 21
murchanas and  eighty-three tanas. But there is no specitic
information as to the number and exact nature of srutis. The
commentary on ``Nanadiyasiksha'' by Acharya Subhankasa belongs to the
12th century. The Bharatiya school that prevails in the South had many
exponents, chief among them being Bahulaka who wrote a karika about
400 A.D., Harsha, Bhatta, Lollata, Sri Sankuka son of Bhatta Mayura,
Rudrata (possibly the Rudracharya of Kudimiyamaiai inseription)
Battodbhata, Sri Ghantika Kohala, Sardula, Durga, Matanga author of
Brihaddesi, Umapati, Nakkakutti etc. Mahamaheswara Abhinavagupta and
Bhatta Nayaka wrote on Abhinaya and expounded the theory of Bharata
about the genesis of Rasa not only in Drisya but also in Sravya
Kavyas.

In Kannada literature we find mention of certain indigenous modes of
poetic composition such as Bedande, Chattana, Madanavati, Chaupadi,
Shatpadi, Gite, Ele, Sangatya, Tivadi, Utsaba, Akkara, Akkarike,
etc. Melvadu, Bajanegabhe and Padugabha Were peculiarly fitted for
purposes of singing. From the times of Western Chalukyas of Kalyani we
have more information to go upon. Sarvagna Someawara was a eultured
sovereign who took great interest in the fine arts of his time. His
work Abhilashitartha Chintamani is a  compendium of all arts from
cookery to kingship. But what is offered to us as specimens of the
kannada songs of the day is now so mutilated by generations of
unintelligent seribes that it is almost impossible to get an idea of
their nature. One or two seem to be in the ordinary kanda metre. But
that Someswara made certain imporvements of his own might be
conjectured, for we are told that ``In Kalyana kataka at the festival
of Bhata matri, actors dressed as Bhillis used to sing and act
Semeswara improved upon it and called the new type Gonditi which from
thence spread to the Maharashtra.'' He was followed in about 1160
A. D. by Vira Bhalluta who wrote his Natya Sastra at the court of
Rudradeva of Warangal.

\begin{center}
\textbf{Sarngadeva}
\end{center}

The next general landmark is the Saugion Ratnakara of Sarngadeva. He
was patronised by Yadava Billama of Sevagiri and seems to have been
the author of another work ``Adhyatma Viveka'' which is no longer
extant. Ratnakara is also called Saptadhyayi as it contains seven
chapters dealing with Svara, Raga. Prakirnaka, Prabandha, Tala, Vadya
and Nartana. Sarngdeva, professes usually to follow Bharata but did
not disdain to barrow from older authors like Kohala Matanga
Nandikeswara, Mahadeva, Kambala Asvatara, etc Raghunatha tells us that
he followed Durga and Dattila also. There are many passages in
Sarngdeva's work which are very diflicult to interpret. He discourses
upon melakartas, Janyaragams and the twelve vikrita svarasthamas but
gives no adequate information about murchanas as employed in his time,
perhaps tranting to oral tradition and practical exemplification of
delanas, gamakas etc. 

Vira Ballala II the Hoysala emperor who had the titles Pratapa
Chakravarti and Sangita prasanga bhangi Bharata was evidently the
author of Bangita Chudamani. The Chalukya Haribhupala wrote his work
``Sangita Sudhakara.'' probably at the same time. In A. D. 1236
Jayasenapati composed his ``Nritta Ratnavali'' at the court of
Ganapati of Warangal. About 1300 A. D. Paswadeva, a Karnataka Jaina,
wrote his ``Sangita Samaya sara.''

\begin{center} 
\textbf{The Vijayanagar Empire}
\end{center}

With the foundation of the Vijayanagara empire a fresh impetus was
given for the development of fine arts. The great Vidyaranya himself
wrote ``Sangita sara'' dealing with 264 ragams and the characteristics
of each. In the time of Bukka Ashtavadhani Somanatha wrote his ``Natya
Chudamani.'' Under Deva Raya II the fine arts were greatly patronised
as I have pointed out in the Indian Antiquary (May 1928). At the
emperor's instance Kallinatha wrote his commentary on Sangita
Ratnakara. Of all the commentaries on this work by Simhabhapa. Kumbhi
Karnauarandra. Hamsalibupa, Gangarams, etc. Kallinatha's alone is
lueid and illuminating.  During the same King's reign, Devarabhatta
wrote his ``Sangita Muktavati, Peddakomativema his ``Sangita
Chinta. ?? ?? ?? ?? ?? dipika.'' Thus we find that music also
deurished to a great extent under that illustrious sovereign.

\begin{center}
\textbf{Later Composers}
\end{center}

After them there follows a hiatus in the history of Karnataka music
and nearly a century passed without pro during writers of great merit
till we come to Pundarika Vithala. This great musician was a native of
Satanus near Sivaganga in the Bangalore district of the Mysore
Provinee and is the author of Ragamanjari, Sadragaebandrodaya,
Ragamala and Nrityanirnaya. he was the contemporary of Tanasena and
Tulsidasa. He wrote his Sadragachandrodaya at the request of Burhana
Khan Pharki, son of Tajakhana and his Ragamanjari under the patronage
of Madhava and Mana Simha, sons of Bhagavandas, at the court of
Emperor Akbar whose qualities are also eulogised in his
Nrityanirnaya. He gives the following equation of Hindustani and
Karnataka Ragas which is of great interest:
\begin{center}
\begin{tabular}{ll}
\textbf{Hindustani}  & \textbf{Karnataka}\\
Kalyani & Mechakalyani\\
Bitaval & Dhirasankarabharana\\
Khamaj & Harikambhoji\\
Bhairav & Mayamalava Gaula\\
Purvi & Kamavardhini\\
Marava & Gamanasri\\
Kafi & Kharaharapriya\\
Asaveri & Natha Bhairavi\\
Bhairavi & Hanumattodi\\
Todi & Subha Pantuvaraji\\
\end{tabular}
\end{center}

Mr. Fox Strangways, however, in his Music of Hindustan gives
Hindustani Sindhubhairavi as equivalent to Natha bhairavi, Jhunjhoti
as equal to Karnataka Harikambeji, and Iman Kalyani as equal to
Kalyani.

Next we come to Ramayamatya Todaramalla (c. 1575 A. D.) author of
Swaramela Kalanidhi. His seales appear to be different from
Bharata's. His Chyata Panchama seems to be a sruti above probably with
a view to the tuning of the Vina. He deseribes twenty mela kartas and
sixty four janya ragas. His rishabha, gandbara, dhaivata and nishadha
tonic seales differ from those of Sarnga deva. His system is styled
Dakshinatya and was greatly in vogue till the rise of Chaturdandi
style. In spite of recriminations of later writers he seems to have
been a better custodian of tradition than those who accuse him of
heretical innovations.

\begin{center}
\textbf{Venkatamakhi}
\end{center}

The next important writer is Somanatha Pandita who in 1609 wrote his
Raga Vibodha with his own commentary. He admits only twenty-two srutis
into his scheme. After him come Raghunatha and Venkatamakhi. The first
was the anther of Sangita Sadha and patronised the latter. He relies
mainly on Sarangadhara, Sangita Sam of Vidyaranya, Sangita Chandrika
of Bhatta Madhava and the commentaries of Kallinatha and Kesara on the
Santadhyayi. Venkatamakhi, author of Chaturdandi, was the son of
Govinda Dikshita and the disciple of Yajnanarayana Dikshita. He
obtained his knowledge of music from Tanappacharya, son of Honaayya,
who, after the destraction of the Empire of Vijayanagara sought
patronaga in other lands. It is important that this faet should be
remembered, because it is the Chaturdandimata of Venkatamakhi that
dispisced to Todaramalla's ``Dakshinatya'' style and was followed by
Tyagaraja and other ister  compesers in the Telugu and Tamil countries
and even now reins supreme in Southern India.

\begin{center}
\textbf{The Moderns}
\end{center}

About 1650 A. D. Abobila Panditha wrote his Sangita Parijata. He
enunciaten 20 srutis and twelve swaras seven pure and five mixed and
for these awaras he lays down the precise length of ??? on the
vina. After him we hear of no great original writer or composer till
the rise of Tyagaraja. Tyagayya is said to have perfected his
knowledge of music from a book of Narada which was presented to him in
person. Under him Karnataka music received its final form.  The
elaberation of the pallavi and anu pallavi, varnams, sangatis,
dolanas, gamakas and a hundred other technical subtieties were defined
and analysed with mathematical precision so that Karnataka music
became a science in the sense of a system ??? of know ledge. In the
??? ??? and other composed ?? ?? ?? ?? Karuri Daks ???? ???,
Muthuswami ??? ??? ??  not become a pedantic display meaningless
forms. It acquired its vitality through Bhakti and incidentally Bhakti
became the sole subject-matter of the kirtanakaras. Perhaps this is as
it should be for songs dealing in all his ?? ?? with the eternal
problems ?? ?? happiness can coneeivably ???  ??  ject-matter more
worth ??? ??? ?? than the relationship ??? ??? the creator and the
creature. They are primarily and necessarily the symbols of the
culture of the people; and no culture can exist without religion and
faith. If the religions element is eliminated, the result will be a
mere civilisation, dead, inert, lifeless as it is in the West to-day.

\begin{center}
\textbf{Popular Developments in West and North}
\end{center}

While thus what might be called a vertical development into the realms
of pure science was going on in the eastern and southern parts of the
Karnataka country, a horizontal expansion in the nature of
popularising the musieal modes was taking place in the western and
northern paris. From the swelfth century onwards the followers of
Basava and Madhwa were aiming at demoeratization of learning. Thus
Sangatya, Sataka, Shatpadi, Tripadi Ragales, Vachanas, etc. were
utilished by Vira Saiva authors while sutadis, kirtanams, devanamams,
chaupadi, etc., became the monopoly of the `Dasakuta. Harikatha
kalakshepams, kelikas and yaksha ganas became powerful factors in the
religious life of the people. Purandaradasa, Vijayadasa, Kanakadasa
and other great composers struck out for themselves a new path in the
field of music. Their compositions, through not so elaborate and
elegant as those of Tyagayya, are yet souched in simple and
spontaneous language give expression to those joys and sorrows that
are near and dear to the heart of every mortal, so that it is not
suprising that many people, disgusted with the intellectual and
physical gymnasties of a songster who mistakes the means for the end
and parodies the soulfal  hymns of Tyagayya, prefer the simple. homely
and chaste beauty of Dasakuta hymns.


\end{document}
