\documentclass{book}

%\font\kan=lelr7t at 12pt

\begin{document}

\chapter*{Campanology in India}

\begin{center}
\textbf{\LARGE Bell-Ringing, an Aid to Worship}\\

\medskip
\large{\textbf{By S. Srikantha Sastri, M.A.}}
\end{center}

Bell-ringing as an aid to worship is an art of extreme antiquity in
India and China where some of the oldest and the biggest bells in the
world are to be found. In Agamic worship, the sound of the bells
indicates the beginning as well as the ending of the ritual. Its
function is to invite the Gods and chase away the evil influences so
that the worship might proceed unioterrupted to the very end when, at
the waving of lights, the union of light and sound is intended to
convey the esoteric significance of the two aspects of Brahman as
tejas and Sabda. Therefore the agamas prescibe elaborata rules as to
the shape, size, and the tuning capacity of the bell and several
deities are supposed to preside over the different parts of the
bell. The small bells used in daily worship are surmounted usually by
the figure of some minor God like Nandi, Hanuman or Garuda, to act as
the mediator between the worshipper and the chief deity. Except
perhaps on the war-chariots, the bell, in India was used little for
secular purposes, and its present shape is evidently of
\textit{tantric} origin. The pinched waist, and the broadened skirt,
it can be conjectured, are the result of the evolution from the
steatopygous figurines of the Mothergoddess, found at Harappa and
Mohenjodaro as well as in other ancient sites through out Asia.

\begin{center}
\textbf{The Bell-Capitol of Mauryan Art}
\end{center}

The Bell-captiol of Mauryan art is now recognised to be of indigenous
origin, developed from the lotus-motif. But the whole pillar
represents nothing but a variety of the usual flag-staff
(Dhvaja-stambha) which was adorned with bells and it is not improbable
that, as the column became more and more stereotyped, the artists
converted the top itself into the shape of a bell which due to the
decorations, assumed the aspect of an inverted lotus. We know for
certain that in the days of Patanjali, the bell along with the other
musical instruments played a prominent part in the worship of the
deities.

The Bell was adopted by the Catholic church in Europe along with other
Eastern ceremonial forms, but as an art, bell-ringing developed late
and though some excellent bells were cast in Italy and Netherlands,
they were not tuned to a definite scale. Campanology as an art dates
from the 16th century in England. Stedman worked out the ``changes''
by permutation and combination according to the number of the bells,
but it was not until the principle of the fivetone harmonic system was
standardised by Canon Simpson, that it was definitely recognised as
both a science and an art. 

In India some of our great temples boast of a few excellent bells,
given by pious donors as gifts in memory of favours received. Two huge
bells attached to the over-hanging precipice at Sivaganga, near
Bangalore, have now disappeared but in the ancient temple on the hill,
are several bells dating from the days of Magadi chiefs. Elsewhere we
find several bells mellow with age but in the modern temples, it is a
sad fact that the bells are so crude and unscientifically tuned that
they defeat, by their discord, the very purpose for which they are
intended. Therefore, since even in the west, no attempt is made to
play hymn-tunes on the  church-bells, but harmony is produced by
ringing the changes, the system can be adapted to suit our conditions.

\begin{center}
\textbf{The English System}
\end{center}

Briefly, the English system is as follows. Three bells give six
distinct changes, four yield 24 changes, five bells give 120 and so on
up to a maximum of 12 bells the greater the number of bells greater
the skill required in jumping ?? that at the close ???  change is
repeated. In the ``Bob ???? six bells are used, in the ``Gra ??
Doubles and Stedman Doubles.'' five balls with the tenor following
behind; in the ``Grandsire and Stedman Triples.'' there are seven
hells with the tener covering; in the Bob Major whole 8 change of
bells, in Caters nines in Cinques and Maki Mus, 12 hells are
employed. The notes are five:-- the \textit{strike},  the \textit{hum}
(an octave below the strike), the \textit{nominal} (an octave above),
the tierce (a correct third above the strike) and the \textit{Quint}
(a fifth above the strike).

Several modifications are necessary before this system can be adopted
to our Karnatic scale. The equal temperamental system is alien to use
and the indigenous system gives us better fifths and thirds than any
other cyclical system. But it is advisable to go back to the older
Kharaharapriya scale as more appropriate than the Mayamalava Gowla or
Sankarabharana of the present day. And tanam varnams can also be
reproduced if slight mechamcal changes are effected. Seven or eight
bells, as in the ``Bob Major'' tuned to our Svara scale can be made to
yield sweet and rich music. The sacred character of the Pranava sound
underlying all the variations in the tune, becomes apparent and
reverberates in the hearts of devotes. It is to be hoped that the huge
\textit{gopurams} of our most important temples will be equipped with
properly tuned bells and this neglected but necessary part of
religious ritual will receive due attention of all lovers of music.

\end{document}
